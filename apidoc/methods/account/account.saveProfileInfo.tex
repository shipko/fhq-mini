\subsection{account.saveProfileInfo}
Позволяет редактировать информацию текущего профиля

Метод требует access\_token

Входные параметры:
\begin{itemize}
  \item \textbf{nick} - Новый никнейм пользователя
  \item \textbf{mail} - Новый адрес электронной почты пользователя\\
  (se-kg: по мне так это надо вынести в отдельный метод, так как операция не тривиальная)
  \item \textbf{json\_data} - Новая инфорамация пользователя типа “О себе”
\end{itemize}

\subsubsection{Example}
\begin{Verbatim}[frame=single]

\end{Verbatim}

\subsection{account.saveProfileInfo - modify}
Позволяет редактировать некоторую информацию текущего профиля

Метод требует access\_token

Входные параметры:
\begin{itemize}
  \item \textbf{param} - название параметра
  \item \textbf{value} - значение параметра
\end{itemize}

\subsubsection{Example}
\begin{Verbatim}[frame=single]

\end{Verbatim}


\subsection{account.changeMail - modify}
Метод требует access\_token
Позволяет редактировать информацию об электронном ящике

1. Отправить инфу пользователю на мыло о подтвержении смены ящика \\
2. И если только пользователь подтвердил то производить смену ящика \\

Мое мнение - это очень опасная функция так как таким образом можно угнать аккаунт. \\
Но функция и полезная вдруг нужно будет сменить мыло. Так что предлагаю эту функцию оставить но только для админа.

Входные параметры:
\begin{itemize}
  \item \textbf{oldmail} - Старый адрес электронной почты пользователя
  \item \textbf{newmail} - Новый адрес электронной почты пользователя
  \item \textbf{pass} - пароль для подтверждения
\end{itemize}

\subsubsection{Example}
\begin{Verbatim}[frame=single]

\end{Verbatim}

\subsection{account.changeNick - modify}
Позволяет изменить никнейм у текущего пользователя

Метод требует access\_token

Входные параметры:
\begin{itemize}
  \item \textbf{nick} - Новый никнейм пользователя
\end{itemize}

\subsubsection{Example}
\begin{Verbatim}[frame=single]

\end{Verbatim}
